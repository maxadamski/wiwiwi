\documentclass[11pt,twocolumn]{article}
\usepackage{lmodern,setspace,amsmath,amssymb,amsfonts,amsthm,graphicx,multicol,grffile,float,csvsimple}
\usepackage[a4paper, top=0.9in, bottom=1.05in, left=1.05in, right=1.05in]{geometry}
\usepackage[polish]{babel}
\usepackage[utf8]{inputenc}
\usepackage[T1]{fontenc}
\title{Algorytmy i struktury danych - Złożone struktury danych}
\author{Dariusz Max Adamski}
\date{}

\begin{document}

\maketitle



\section*{Wstęp}

W tym sprawozdaniu porównywane będą czasy wykonywania czynności dodawania, wyszukiwania oraz usuwania elementów, w liście jednokierunkowej, drzewie poszukiwań binarnych BST i w dokładnie wyważonym BST.



\section*{Metodologia}

Pomiary wykonywane były na tablicach o wielkości $n$ od $1\ 000$ elementów do $20\ 000$ elementów, z krokiem $1\ 000$ (20 punktów pomiarowych).

Uprzednio wygenerowano losowo listę $50\ 000$ studentów (elementy uporządkowane rosnąco według indeksu). Przed rozpoczęciem pomiarów lista została załadowana do pamięci, i zapisana jako tablica $S$ wskaźników $Student*$. W każdej iteracji wykonywane są poniższe kroki:

\begin{enumerate}
	\item Tworzone jest $P := permutacja(0..n-1)$.
	\item Tworzona jest struktura danych $D$.
	\item Dla każdego $i$ od $0$ do $n-1$, student $S_i$ jest dodawany do $D$.
	\item Dla każdego $i$ w $P$ student $S_i$ jest wyszukiwany w $D$.
	\item Dla każdego $i$ od $n-1$ do $0$ student $S_i$ jest usuwany z $D$.
	\item Kroki 2 do 5 są powtarzane dla każdej testowanej struktury danych.
\end{enumerate}

Czasy wykonywania kroku dodawania (append) i usuwania (remove) są przedstawione w sekundach, a czasy wykonywania kroku znajdywania (find) zostały podzielone przez $n$, i są przedstawione w milisekundach.

Aby utworzyć drzewo BBST, do struktury dodawana jest najpierw mediana $D_{0..n-1}$, a później rekurencyjnie mediany lewej i prawej podlist $D$, wydzielonych przez medianę.

Aby zbadać przypadek średni BST, do struktury elementy były dodawane w losowej kolejności. Ten przypadek został oznaczony ,,bst-rand''.

Optymalizacja kompilatora została wyłączona flagą ,,-O0''. Czas wykonywania był mierzony w nanosekundach.



\section{Dodawanie}

\begin{figure}[h]
	\includegraphics[width=\linewidth]{../data/append.png}
	\caption{Czas dodawania $n$ elementów \label{append}}
\end{figure}

\begin{table}[h]
	\centering
	\csvautotabular{../data/append.csv}
	\caption{Czas dodawania $n$ elementów (s)}
\end{table}

Czynność dodawania $n$ elementów ma złożoność $O(n^2)$ dla listy i BST. W losowym BST (bst-rand) złożoność to $O(\log n)$.

Operacja dodawania elementu do BST ma złożoności $O(n)$, ponieważ elementy dodawane są według indeksów rosnąco (przypadek zdegenerowany). Dodając elementy losowo, w średnim przypadku operacja ma złożoność $O(\log n)$.

Dla listy dodawanie (na koniec) także ma złożoność $O(n)$ ponieważ nie jest przechowywany wskaźnik do ostatniego elementu listy, więc trzeba przejść przez elementy w liście. Przy zachowanym wskaźniku dodawanie na koniec ma złożoność $O(1)$.

Zdegenerowane BST w każdym kroku wykonuje dwa porównania, więc czas dodawania jest około 2 razy większy od dodawania do listy.



\section{Usuwanie}

\begin{table}[h!]
	\centering
	\csvautotabular{../data/remove.csv}
	\caption{Czas usuwania $n$ elementów (s)}
\end{table}

Usuwanie $n$ elementów, dla testowanych struktur danych, ma taką samą złożoność jak dodawanie.

Usuwanie z listy także ma złożoność $O(n)$. Tym razem operacja może zakończyć się przed przejściem całej listy.

Podobnie jak w dodawaniu możemy zastosować trik. Usuwanie pierwszego elementu ma złożoność $O(1)$, więc jeżeli chcemy usunąć wszystkie elementy, złożoność tej operacji będzie $O(n)$.

Czasy usuwania są większe od dodawania prawdopodobnie przez potrzebę uwalniania pamięci.



\section{Znajdywanie}

\begin{figure}[h!]
	\includegraphics[width=\linewidth]{../data/find.png}
	\caption{Średni czas znajdywania elementu \label{find}}
\end{figure}

Znajdywanie elementu ma złożoność $O(n)$ w liście i zdegenerowanym BST, a w losowym BST oraz BBST $O(\log n)$. Zdegenerowane drzewo BST musi wykonywać więcej operacji porównań, więc jest wolniejsze od listy.

Losowe BST jest wolniejsze od drzewa BBST średnio o parę mikrosekund. Czas dostępu jest nadal bardzo dobry.

BBST bardzo szybko znajduje studentów. Dla $20\ 000$ elementów w drzewie potrzebne jest maksymalnie 15 kroków aby dotrzeć do jakiegokolwiek z elementów, więc zajmuje to tylko kilka mikrosekund.

Nawet dla $10\ 000\ 000$ studentów, czyli największej ilości dozwolonej przez format indeksu, wystarczą 24 kroki na znalezienie dowolnego studenta!

\begin{table}[h]
	\centering
	\resizebox{\columnwidth}{!}{
		\csvautotabular{../data/find.csv}
	}
	\caption{Średni czas znajdywania elementu (ms)}
\end{table}



\section{Wnioski}

Podsumowując, opłaca się używać drzewa BST, jeżeli możemy sobie pozwolić na bardziej skomplikowany algorytm i jesteśmy gotowi je balansować.

W przypadku danych posortowanych, drzewo BST jest wolniejsze w każdej kategorii od listy jednokierunkowej.

Ponadto gdy wiemy jaki jest ostatni element listy, dodanie kolejnego elementu do listy jest mniej kosztowne niż dodanie go do BST, nawet gdy jest ono zbalansowane (ale $\log n$ to nie jest tak dużo).

Alternatywnie, lista jednokierunkowa stanowi bardzo prosty zamiennik, który jest wystarczający dla większości zastosowań.

Jeżeli jednak mamy wszystkie dane dostępne (możemy je zapisać w losowej kolejności) i chcemy je zapisać w strukturze danych, drzewo BST jest dobrym wyborem.



\end{document}

