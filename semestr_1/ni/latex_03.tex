\documentclass{beamer}
\usepackage{lmodern,setspace,amsmath,amssymb,amsfonts,amsthm}
\usepackage[polish]{babel}
\usepackage[utf8]{inputenc}
\usepackage[T1]{fontenc}
\usetheme{Warsaw}
\title{Tytuł: Matematyka}
\author{Ktoś}
\institute{PP II}
\date{Poznań, \today}

\begin{document}

\maketitle


\section{Ściąga}

\subsection{preambuła}
\begin{frame}
	\frametitle{preambuła}
\end{frame}

\subsection{strona tytułowa}
\begin{frame}
	\frametitle{strona tytułowa}
\end{frame}


\section{Ściąga jak zrobić slajdy}

\subsection{źródło jednego ze slajdów}
\begin{frame}
	\frametitle{praca domowa}
	\begin{block}{samodzielna}
		wykonać pozostałe slajdy
	\end{block}
	\begin{block}{}
		jakiś kod
	\end{block}
\end{frame}

\subsection{slajd bez tytułu slajdu}
\begin{frame}
	int,lim,frac,sum,lnot,land,implies,Rightarrow,infty,iff,nearrow
\end{frame}

\subsection{slajd z tytułem slajdu}
\begin{frame}
	\frametitle{tytuł1}
	int,lim,frac,sum,lnot,land,implies,Rightarrow,infty,iff,nearrow
\end{frame}

\subsection{slajd z tytułem, z blokiem bez tytułu}
\begin{frame}
	\frametitle{tytuł2}
	\begin{block}{}
		int,lim,frac,sum,lnot,land,implies,Rightarrow,infty,iff,nearrow
	\end{block}
\end{frame}

\subsection{slajd z tytułem, z blokiem z tytułem}
\begin{frame}
	\frametitle{tytuł3}
	\begin{block}{tytuł bloku}
		int,lim,frac,sum,lnot,land,implies,Rightarrow,infty,iff,nearrow
	\end{block}
\end{frame}


\section{Trygonometria (array)}

\begin{frame}
	\frametitle{tabelka}

	\begin{spacing}{1,5}
	\[
		\begin{array}{|c|c|c|c|c|c||c|c|c|c|}
			\hline
			x & 0 & \frac{\pi}{6} & \frac{\pi}{4} & \frac{\pi}{3} & \frac{\pi}{2} &
			(0; \frac{\pi}{2}) & (\frac{\pi}{2};\pi) & (\pi;\frac{3\pi}{2}) & (\frac{3\pi}{2}; 2\pi)\\
			\hline
			\sin x & 0 & \frac{1}{2} & \frac{\sqrt{2}}{2} & \frac{\sqrt{3}}{2} & 1 & + & + & - & -\\
			\hline
		\end{array}
	\]
	\end{spacing}
\end{frame}


\section{Szeregi (description)}

\begin{frame}
	\frametitle{Szeregi}

	\begin{description}
		\item[Szereg nineskończony:] $\displaystyle \sum_{n=1}^\infty a_n = (a_n, S_n)$
		\item[Szereg jest zbieżny:] $\displaystyle \sum_{n=1}^\infty a_n < \infty$
		\item[Szereg jest rozbieżny:] $\displaystyle \sum_{n=1}^\infty a_n = \infty$
	\end{description}

	\noindent Warunek konieczny zbieżności: $\displaystyle \sum_{n=1}^\infty a_n < \infty \Rightarrow \lim_{n \to \infty} a_n = 0$ \\

\end{frame}

\subsection{Szeregi o wyrazach dodatnich (equation, tag)}

\begin{frame}
	\frametitle{Szeregi o wyrazach dodatnich}
	\begin{block}{Funkcja Riemanna}
		$$
			\zeta (s) = \sum_{n=1}^\infty \frac{1}{n^s} = 
				\begin{cases}
					\infty & \text{dla } s \leq 1 \\
					-\infty & \text{dla } s \ge 1
				\end{cases}
		$$
	\end{block}

\end{frame}

\subsection{Szeregi o wyrazach dowolnych (align)}

\begin{frame}
	\frametitle{Szeregi o wyrazach dowolnych}
	\begin{block}{Kryterium Abela:}
		$$
		\left(\lnot [(a_n) \nearrow] \land \forall{n \in N} : (a_n 0) \land \sum_{n=1}^\infty b_n < \infty \right) \Rightarrow \sum_{n=1}^\infty (a_n b_n) < \infty
		$$
	\end{block}


	\begin{block}{Kryterium Dirichleta:}
		$$
		\left(\lim_{n \to \infty} a_n = 0 \land \lnot [(a_n) \nearrow] \land \exists{\epsilon > 0} : \forall{n \in N} : \epsilon - |S_n| \right) \\
		\hfill \Rightarrow \displaystyle \sum_{n=1}^\infty (a_n b_n) < \infty
		$$
	\end{block}
\end{frame}


\section{Całki (gather,int)}

\begin{frame}
	\frametitle{Całki}
	hmmm co to są całki?
\end{frame}


\section{Spis treści}

\begin{frame}
	\tableofcontents
\end{frame}

\end{document}
