\documentclass[11pt]{article}
\usepackage{setspace,lmodern,amsmath,amssymb,amsfonts,amsthm}
\usepackage[top=1in, bottom=1.25in, left=1.25in, right=1.25in]{geometry}
\usepackage[polish]{babel}
\usepackage[utf8]{inputenc}
\date{}
\setlength{\parindent}{0cm}

\begin{document}

\begin{center}
	\LARGE Podstawy Automatyki - Sprawdzian 1 \\
\end{center}
\vspace{0.4cm}

\section{Człony}

k - współczynnik wzmocnienia\\
T - stała czasowa inercji / całkowania / opóźnienie\\
$\xi$ - współczynnik tłumienia $\in (0, 1)$\\
$\omega_0$ - pulsacja oscylacji własnych


\subsection{Obiekty proporcjonalne}

\subsubsection{Obiekt wzmacniający idealny}

$y(t) = k x(t)$\\

$G(s) = k$\\

\textbf{Zastosowania:} wzmacniacz bezinercyjny, maszyny proste

\subsubsection{Obiekt wzmacniający rzeczywisty}

$T \frac{dy(t)}{dt} + y(t) = k x(t)$\\

$G(s) = \frac{k}{1 + sT}$\\

\textbf{Zastosowania:} wzmacniacz rzeczywisty, maszyny proste, zawór

\subsubsection{Obiekt inercyjny n-tego rzędu}

$
T_1 T_2 ... T_n \frac{d^n y(t)}{dt^n} + ...
+ (T_1 + T_2 + ... T_n) \frac{d y(t)}{dt} + y(t) =
k x(t)
$\\

$G(s) = \frac{k}{(1 + sT_1)(1 + sT_2)...(1 + sT_n)}$\\

\textbf{Zastosowania drugiego rzędu:} maszyny proste, zawory z niekorzystnymi zjawiskami\\
\textbf{Zastosowania n-tego rzędu:} złożone układy hydrauliczne, mechaniczne i elektryczne

\subsection{Obiekt oscylacyjny}

$
T^2 \frac{d^2 y(t)}{dt^2} +
2 \xi T \frac{dy(t)}{dt} + y(t) =
k x(t)
$\\

$G(s) = \frac{k \omega_0^2 }{s^2 + 2 \xi \omega_0 s + \omega_0^2}$\\

\textbf{Zastosowania:} układy mechaniczne oscylujące (masa i sprężyna), elektryczny układ drgający, wahadło

\subsection{Obiekty różniczkujące}

\subsubsection{Obiekt różniczkujący idealny}

$y(t) = k \frac{dx(t)}{dt}$\\

$G(s) = ks$\\

\textbf{Zastosowania:} brak

\subsubsection{Obiekt różniczkujący rzeczywisty}

$T \frac{dy(t)}{dt} + y(t) = k \frac{dx(t)}{dt}$\\

$G(s) = \frac{ks}{1 + sT}$\\

\textbf{Zastosowania:} cewka indukcyjna, tłumik hydrauliczny, tarcie mechaniczne


\subsection{Obiekty całkujące}

\subsubsection{Obiekt całkujący idealny}

$y(t) = k \int_0^t x(\tau) d\tau$\\

$G(s) = \frac{k}{s}$\\

\textbf{Zastosowania:} kondensator idealny

\subsubsection{Obiekt całkujący rzeczywisty}

$T \frac{dy(t)}{dt} + y(t) = k \int_0^t x(\tau) d\tau$\\

$G(s) = \frac{k}{s(1 + sT)}$\\

\textbf{Zastosowania:} kondensator, zbiornik cieczy


\subsection{Obiekt opóźniający}

$y(t) = x(t-T)$\\

$G(s) = e^{-sT}$\\

\textbf{Zastosowania:} transporter taśmowy



\pagebreak

\section{Definicje}

\subsection{Transmitancja operatorowa}

$$G(s) = \frac{Y(s)}{X(s)}$$

Stosunek transformaty Laplace'a odpowiedzi do transformaty Laplace'a wymuszenia, przy zerowych warunkach początkowych.


\subsection{Charakterystyka czasowa}

Przebieg czasowy wyjścia, wywołany wymuszeniem.

\subsection{Transmitancja widmowa}

Stosuek zespolonej składowej odpowiedzi do zespolonej składowej wymuszenia sinusoidalnego. Wykres transmitancji widmowej na płaszczyźnie Gaussa nazywamy charakterystyką amplitudowo-fazową.

$$
G(j\omega) = \frac{\hat Y_W}{\hat X} =
\frac{A_X e^{j \omega t}}{A_{Y_W} e^{j(\omega t + \phi)}} =
P(\omega) + jQ(\omega)
$$


\section{Charakterystyki}

\subsection{Charakterystyka impulsowa}

\[
	\delta(t) = g(t) = \begin{cases}
		\infty, & \text{dla } t = 0 \\
		0, & \text{dla } t \neq 0 \\
	\end{cases}
	\text{ oraz }
	X(s) = 1
\]

\subsection{Charakterystyka skokowa}

\[
	\textbf{1}(t) = h(t) = \begin{cases}
		1, & \text{dla } t \geq 0 \\
		0, & \text{dla } t < 0 \\
	\end{cases}
	\text{ oraz }
	X(s) = \frac{1}{s}
\]

\subsection{Logarytmiczna charakterystyka amplitudowa}

\[
	\frac{|Y|}{|X|} = \sqrt[20]{10}
	\rightarrow
	20 \log \frac{|Y|}{|X|} = \log 10 = 1
\]



\end{document}
