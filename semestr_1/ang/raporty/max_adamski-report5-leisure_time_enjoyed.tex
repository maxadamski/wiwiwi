\documentclass[12pt]{article}
\usepackage{lmodern}
\usepackage[top=1in, bottom=1.25in, left=1.25in, right=1.25in]{geometry}
\usepackage[utf8]{inputenc}
\usepackage[T1]{fontenc}
\usepackage[parfill]{parskip}

\begin{document}
\pagenumbering{gobble}

\begin{flushleft} 
	Dariusz Max Adamski \\
	Grupa 2 Informatyka WI\\
	Nr indeksu 136674
\end{flushleft}

\begin{center} 
	\vspace{0.8cm} \Large 
	Leisure Time Enjoyed by Men and Women of Different Employment Status
	\vspace{0.5cm}
\end{center}

The chart shows the amount of leisure time enjoyed by men and women of different employment status, in a typical week, during the period of 1998 to 1999.

Unemployed men and women had 85 and 78 hours of free time respectively, which was the same as retired men and women.

Men and women employed full-time had 45 and 35 hours of leisure time respectively, while women employed part-time had 40 hours, and housewives had 55 hours. The amount of leisure time was not measured for men employed part-time.

Unemployed and retired men and women had the most hours of leisure time in a typical week, while housewives, people employed full-time and people employed part-time had the least.

In summary, unemployed and retired people had twice as much free time as people employed full-time or part-time. Where measured, men had 10 hours of leisure time more, in a typical week, than women of the same employment status. 

156 words

\end{document}
