\documentclass[12pt]{article}
\usepackage{lmodern}
\usepackage[top=1in, bottom=1.25in, left=1.25in, right=1.25in]{geometry}
\usepackage[utf8]{inputenc}
\usepackage[T1]{fontenc}
\usepackage[parfill]{parskip}

\begin{document}

\begin{flushleft} 
	Dariusz Max Adamski \\
	Grupa 2 Informatyka WI\\
	Nr indeksu 136674
\end{flushleft}

\begin{center} 
	\vspace{0.8cm} \Large
	Film review - Doogtoth
	\vspace{0.5cm}
\end{center}

Dogtooth is a 2009 Greek drama film directed by Yorgos Lanthimos. The film focuses on a family living on the outskirts of an unknown city. The overprotective parents homeschool their three adult children, keeping them isolated from the outside world and it's bad influences. Every day the children participate in family life activities. They train, play and learn. Stickers are awarded for good behavior, but bad behavior is penalized in an almost sadistic manner.

Meanwhile, the parents do everything they can to keep their children from leaving the house. For example, according to them, leaving the premises results in certain death, and the only way to avoid it is to drive a car (which only the father can do). The lies they tell are so absurd, yet the gullible children believe in everything their parents tell them.

Disorder enters the house when Christina, a woman working as a security guard at the father's business, is employed to tame the sexual urges of the son. One day she gives the eldest daughter a headband with sparkling stones, and later VHS tapes of "Rocky" and "Jaws", unintentionally planting questions about the outside world in the daughter's mind.

On the technical side, the minimalist cinematography, and the soundtrack, or rather it's absence, works wonders for the film. These allow Mr. Lanthimos to bring surprising amounts of the elements of suspense and horror into the film. As for acting, it is unlike anything I have ever seen (except for the other works of Mr. Lanthimos, of course). People in Dogtooth say and do things in a manner that no human would ever do, almost like aliens in human bodies. It makes the film so much more eerie and mystical.

A minor complaint is that Dogtooth provides absolutely no answers to questions which arise, so you will have to think by yourself, about why the family ended up this way, or why has not anybody done anything about it. In spite of that, the film is not too elitist about it's ideas, so it is very accessible, even to less demanding viewers.

\pagebreak

In general, I think that Dogtooth is amazing, both as a standalone film, and as a great introduction into the unique filmography of the Greek director. The overprotective family trope, that Mr. Lanthimos plays with, is sure to resonate with most people. Finally, Dogtooth is perplexing, but in the end it makes you look at the human society from a new perspective.

409 words

\end{document}
