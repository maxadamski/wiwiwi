\documentclass[12pt]{article}
\usepackage{lmodern}
\usepackage[top=1in, bottom=1.25in, left=1.25in, right=1.25in]{geometry}
\usepackage[utf8]{inputenc}
\usepackage[T1]{fontenc}
\usepackage[parfill]{parskip}
\pagenumbering{gobble}

\begin{document}

\begin{flushleft} 
	Dariusz Max Adamski \\
	Grupa 2 Informatyka WI\\
	Nr indeksu 136674
\end{flushleft}

\begin{center} 
	\vspace{0.8cm} \Large
	Figures for Imprisonment in Different Countries
	\vspace{0.5cm}
\end{center}

The graph shows the figures for imprisonment, grouped by country, in the period of 1930 to 1980.

The number of people imprisoned in Great Britain remained constant at 30 thousand from 1930 to 1940, but steadily rose to 90 thousand in 1980.

In Australia, during the period of 1930 to 1950, there was a 15 thousand decrease, from 70 thousand to 55 thousand, in the number of people imprisoned, but from 1950 to 1970 it rose from 55 thousand to 80 thousand. Finally it fell to 50 thousand in 1980.

Similarly, in New Zealand, during the period of 1930 to 1950, the number of people of people imprisoned fell from 100 thousand to 58 thousand. However, it then steadily rose to 92 thousand in 1980.

In the United States the figures for imprisonment sharply rose from 100 thousand in 1930 to 130 thousand in 1940, then decreased to 100 thousand in 1970, only to peak at 140 thousand in 1980.

The number of people imprisoned in Canada had been steadily decreasing from 120 thousand in 1930 to 100 thousand in 1980.

In summary, the figures for imprisonment decreased in Australia, New Zealand and Canada, but increased in Great Britain and United States.

202 words

\end{document}
