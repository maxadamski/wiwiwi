\documentclass[12pt]{article}
\usepackage{lmodern}
\usepackage[top=1in, bottom=1.25in, left=1.25in, right=1.25in]{geometry}
\usepackage[utf8]{inputenc}
\usepackage[T1]{fontenc}
\usepackage[parfill]{parskip}

\begin{document}
\pagenumbering{gobble}

\begin{flushleft} 
	Dariusz Max Adamski \\
	Grupa 2 Informatyka WI\\
	Nr indeksu 136674
\end{flushleft}

\begin{center} 
	\vspace{0.8cm} \Large
	Purposes and Destinations of Visits Abroad \\ of UK Residents
	\vspace{0.5cm}
\end{center}

The first table shows the purspose of visits abroad of UK residents, during the period of 1994 to 1998. Holidays were UK residents's most popular reason for travelling abroad. The number of people travling was 15,246 in 1994, rising to 20,700 in 1998. The number of business trips increased from 3,155 in 1994 to 3,957 in 1998. Visits to friends and relatives also increased from 2,689 in 1994 to 3,181 in 1998. Other reasons were the least popular purpose of visits. Still, it rose from 982 in 1994 to 1,054 in 1996, but then fell to 990 visits in 1998.

The second table shows the destinations of visits abroad of UK residents, during the period of 1994 to 1998. Most UK residents were travelling to Western Europe. The number of people travelling there was 19,371 in 1994, it then steadily rose to 24,519 in 1998. Other areas were the second most popular destination with 1,782 visits in 1994, rising to 2,486 visits in 1998. North America was the least popular destination, with only 919 residents choosig it in 1994. The number then rose to 1,823 in 1998.

In general, the total number of visits abroad increased over the period of 5 years. In consequence, the popularity of almost all destinations and purposes of visits increased. It only decreased slightly in 1995, sharply rising afterwards.

224 words

\end{document}
