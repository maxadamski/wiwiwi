\documentclass[12pt]{article}
\usepackage{lmodern,setspace,amsmath,amssymb,amsfonts,amsthm}
\usepackage[polish]{babel}
\usepackage[utf8]{inputenc}
\usepackage[T1]{fontenc}
\title{Tytuł: Matematyka}
\date{}

\begin{document}

\maketitle

\section*{Ściąga}

int,lim,frac,sum,lnot,land,implies,Rightarrow,infty,iff,nearrow

\section{Trygonometria (array)}

\begin{spacing}{1,5}
\[
	\begin{array}{|c|c|c|c|c|c||c|c|c|c|}
		\hline
		x & 0 & \frac{\pi}{6} & \frac{\pi}{4} & \frac{\pi}{3} & \frac{\pi}{2} &
		(0; \frac{\pi}{2}) & (\frac{\pi}{2};\pi) & (\pi;\frac{3\pi}{2}) & (\frac{3\pi}{2}; 2\pi)\\
		\hline
		\sin x & 0 & \frac{1}{2} & \frac{\sqrt{2}}{2} & \frac{\sqrt{3}}{2} & 1 & + & + & - & -\\
		\hline
	\end{array}
\]
\end{spacing}

\section{Szeregi (description)}

\begin{description}
	\item[Szereg nineskończony:] $\displaystyle \sum_{n=1}^\infty a_n = (a_n, S_n)$
	\item[Szereg jest zbieżny:] $\displaystyle \sum_{n=1}^\infty a_n < \infty$
	\item[Szereg jest rozbieżny do nieskończoności:] $\displaystyle \sum_{n=1}^\infty a_n = \infty$
\end{description}

\noindent Warunek konieczny zbieżności: $\sum_{n=1}^\infty a_n < \infty \Rightarrow \lim_{n \to \infty} a_n = 0$ \\
(wniosek: $\lim_{n \to \infty} a_n \neq 0 \Rightarrow \displaystyle \sum_{n=1}^\infty a_n = \infty$) \\
$\displaystyle \sum_{n=1}^\infty a_n < \infty$ jest bezwględnie zbieżny, 
jeżeli $\displaystyle \sum_{n=1}^\infty |a_n| < \infty$, 
w przeciwnym przypadku jest warunkowo zbieżny.

\subsection{Szeregi o wyrazach dodatnich (equation, tag)}

\subsubsection{Funkcja Riemanna: (cases,text)}

\begin{equation}
	\tag{0}
	\zeta (s) = \sum_{n=1}^\infty \frac{1}{n^s} = 
		\begin{cases}
			\infty & \text{dla } s \leq 1 \\
			-\infty & \text{dla } s \ge 1
		\end{cases}
\end{equation}

\subsection{Szeregi o wyrazach dowolnych (align)}

\subsubsection{Kryterium Abela:}

\begin{equation}
	\left(\lnot [(a_n) \nearrow] \land \forall{n \in N} : (a_n 0) \land \sum_{n=1}^\infty b_n < \infty \right) \Rightarrow \sum_{n=1}^\infty (a_n b_n) < \infty
\end{equation}

\subsubsection{Kryterium Dirichleta:}

\begin{equation}
	\left(\lim_{n \to \infty} a_n = 0 \land \lnot [(a_n) \nearrow] \land \exists{\epsilon > 0} : \forall{n \in N} : \epsilon - |S_n| \right) \Rightarrow \sum_{n=1}^\infty (a_n b_n) < \infty
\end{equation}

\subsubsection{Kryterium Leibniza:}

\begin{equation}
	\lim_{n \to \infty} a_n = 0 \iff \sum_{n=1}^\infty ((-1)^n a_n) < \infty
\end{equation}

\section{Całki (gather,int)}

\begin{equation}
	\int \frac{f'(x)}{f(x)} dx = \ln{|f(x)|}
\end{equation}
\begin{equation}
	\int x dx = x^2 + C
\end{equation}
\begin{equation}
	\int \cos x dx = \sin x + C
\end{equation}

\section{Kombinatoryka (align*, \{a \textbackslash choose b\})}

$${n \choose k} = \frac{n!}{(n-k)!k!}$$
$${n \choose n - k} = {n \choose k}$$

\tableofcontents

\end{document}
