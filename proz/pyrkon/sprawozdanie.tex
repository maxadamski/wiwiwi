\documentclass[11pt]{article}
\usepackage{lmodern,setspace,amsmath,amssymb,amsfonts,amsthm,graphicx,multicol,grffile,float}
\usepackage{authblk,url,csvsimple,cuted,dblfloatfix,parskip}
\usepackage[a4paper, top=0.9in, bottom=1.05in, left=1.01in, right=1.01in]{geometry}
\usepackage[polish]{babel}
\usepackage[utf8]{inputenc}
\usepackage[unicode]{hyperref}
\usepackage{listings}
\usepackage{booktabs}
\usepackage{fancyvrb}
\title{Programowanie Rozproszone\\ Projekt -- Obsługa Pyrkonu}
\author{Dariusz Max Adamski 136674, Sławomir Gilewski 142192}
\affil{\{dariusz.adamski,slawomir.gilewski\}@student.put.poznan.pl}
\date{Data oddania: \today}

\def\code#1{\texttt{#1}}

\begin{document}

\maketitle

%\section*{Wstęp}

\section{Struktury danych}

\subsection{Wiadomość}

Wiadomość składa się z zegara, numeru kolejki i jednej z poniższych etykiet:

\begin{enumerate}
    \item REQ -- Wysyłana z wiadomościami z ubieganiem się o zasób
    \item OK  -- Wysyłana jako odpowiedź na ubieganie się o zasób
    \item END -- Wysyłana w celu zasygnalizowania wyjścia z wydarzenia (numer kolejki ignorowany)
\end{enumerate}

\subsection{Stałe}

\begin{enumerate}
    \item num-tickets -- liczba biletów na Pyrkon
    \item num-workshops -- liczba warsztatów na Pyrkonie
    \item max-places -- liczba miejsc w warsztacie
    \item max-workshops -- liczba warsztatów które chce zwiedzić uczestnik
    \item workshops -- numery warsztatów 1...num-workshops
    \item num-proc -- liczba procesów
\end{enumerate}

\subsection{Zmienne}

\begin{enumerate}
    \item executing -- num-workshops + 1 wortości logicznych, i-ta wartość jest prawdziwa jeśli proces jest w i-tej sekcji krytycznej
    \item requesting -- num-workshops + 1 wartości logicznych, i-ta wartość jest prawdziwa jeśli proces ubiega się o dostęp do i-tej sekcji krytycznej
    \item accepted -- num-workshops + 1 liczb naturalnych, i-ta wartość oznacza liczbę procesów pozwalających procesowi na dostęp do i-tej sekcji
    \item deferred -- num-workshops + 1 liczb procesów do których należy wysłać wiadomość po wyjściu z i-tej sekcji krytycznej
    \item exited -- liczba naturalna oznaczająca ilość procesów, które wyszły z wydarzenia
    \item time -- zegar danego procesu
    \item pid -- ranga procesu
\end{enumerate}


\subsection{Zegar}

Do rozwiązania problemu dostępu do sekcji krytycznej używamy algorytmu Ricarta-Agrawala.

Każdy proces posiada jeden zegar ,,time'' i dołącza go do każdej wysyłanej wiadomości.

Po odebraniu wiadomości zegar jest aktualizowany.

Procedura aktualizacji zegara wygląda następująco:

\begin{Verbatim}[numbers=left,xleftmargin=5mm]
function tick(time'):
    time = max(time, time') + 1
\end{Verbatim}

\section{Obsługa wiadomości}

Zadaniem procedury ACK jest reagowanie na otrzymane wiadomości.
Opcjonalnym parametrem procedury jest req-time, czyli czas zegara,
który zawierała wiadomość REQ wysłana w procedurze P podczas ubiegania się o dostęp.

Najpierw odbieramy wiadomość w trybie nieblokującym.
Jeśli nie otrzymaliśmy żadnej wiadomości, wychodzimy.
Struktura wiadomości zawiera pola z etykietą, zegarem procesu i numerem sekcji krytycznej.

Aktualizujemy lokalny zegar, biorąc pod uwagę zegar z wiadomości.

Następnie, jeśli etykieta wiadomości to REQ, zgodnie z algorytmem Rickarta-Agrawala,
jeśli nie ubiegamy się o dostęp do q-tej sekcji krytycznej ani w niej nie jesteśmy,
lub ubiegamy się o dostęp i nasz proces ma mniejszy priorytet,
to wysyłamy odpowiedź OK.

W przeciwnym przypadku, dodajemy nadawcę do listy deferred[q],
po wyjściu z q-tej sekcji krytycznej wyślemy do wszystkich procesów z tej listy wiadomość OK.

Dla ścisłości, nasz proces ma mniejszy priorytet, kiedy czas zegara wysłany z wiadomością REQ, czyli req-time,
jest późniejszy (większy) od czasu zegara z wiadomości przychodzącej.
W przypadku równych czasów, proces z niższym numerem ma większy priorytet. 

Gdy etykieta wiadomości to OK, zwiększamy accepted[q],
czyli liczbę procesów zgadzających się na nasze wejście do sekcji q-tej krytycznej.

Jeśli etykieta to END, zwiększamy exited,
czyli liczbę procesów które wyszły z pyrkonu.

\begin{Verbatim}[numbers=left,xleftmargin=5mm]
function ACK(req-time = -1):
    if not recv (tag t q) from pid'
        return
    tick(t)
    if tag == REQ:
        lower-priority = pid' < pid if t == time else t < req-time
        if (not requesting[q] and not executing[q])
                or (requesting[q] and lower-priority):
            time += 1
            send (OK time q) to pid'
        else:
            deferred[q].append(pid')
    if tag == OK:
        accepted[q] += 1
    if tag == END:
        exited += 1
\end{Verbatim}

\section{Dostęp do zasobu}

Do obsługi dostępu do sekcji krytycznych stworzyliśmy dwie ogólne procedury: P i V.

Aby ubiegać się o dostęp do q-tego zasobu o pojemności n, należy wywołać procedurę $P(q, n)$.

Procedura na początku wysyła do wszystkich procesów (poza sobą) wiadomość z etykietą REQ, swoim aktualym zegarem i indeksem zasobu q.
Zapisywany jest przy tym zegar w zmiennej req-time.

Następnie ustawiany jest stan procesu, w odniesieniu do q-tego zasobu i do czasu gdy num-proc - n procesów nie zaakceptuje dostępu do zasobu,
odpowadamy na wiadomości w ACK.

Po uzyskaniu akceptacji, aktualizujemy stan.

\begin{Verbatim}[numbers=left,xleftmargin=5mm]
function P(q, n):
    time += 1
    req-time = time
    send (REQ time q) to all except me
    
    requesting[q] = true
    accepted[q] = 0
    
    while accepted[q] < num-proc - n:
        ACK(req-time)
        
    requesting[q] = false
    executing[q] = true
    
\end{Verbatim}

Aby zwolnić dostęp do q-tego zasobu, wywołujemy procedurę $V(q)$.

Procodura wysyła wiadomość OK do wszystkich procesów z q-tej listy deferred[q],
a następnie czyści tą listę i zaznacza, że proces nie wykonuje tej sekcji.

\begin{Verbatim}[numbers=left,xleftmargin=5mm]
function V(q):
    time += 1
    send (OK time q) to deferred[q]
    deferred[q] = []
    executing[q] = false
\end{Verbatim}

\section{Oczekiwanie na zakończenie}

Procedura JOIN służy do oczekiwania na wyjście wszystkich procesów z wydarzenia.

Zaznaczamy fakt, że wyszliśmy z wydarzenia zwiększając liczbę procesów które wyszły.
Następnie wysyłamy do wszystkich (poza sobą) procesów wiadomość END (numer zasobu nie ma znaczenia) .
Gdy proces otrzymuje wiadomość END, zwiększa licznik exited o 1.
Po otrzymaniu wiadomości END od wszystkich procesów,
proces może brać udział w kolejnym pyrkonie.

\begin{Verbatim}[numbers=left,xleftmargin=5mm]
function JOIN():
    exited += 1
    time += 1
    send (END time -1) to all except me
    while exited < num-proc:
        ACK()
\end{Verbatim}

\section{Program główny}

Mając omówione wszystkie zmienne i podprogramy, możemy przejść do opisu programu głównego dla ludzi odwiedzających Pyrkon.

Na początku każdy ubiega się o dostęp do wydarzenia wywołując procedurę P, z numerem kolejki wydarzenia -- 0 i uczciwie podając maksymalną liczbę biletów -- num-tickets.

Następnie uczestnik losuje max-worshops warsztatów z całej puli, a następnie dla każdego wylosowanego warsztatu, podobnie jak przy Pyrkonie, ubiega się o miejsce podając do procedury P numer warsztatu jako numer kolejki i maksymalną liczbę miejsc na tym warsztacie (może być ta sama dla wszystkich, albo inna dla każdego).

Podczas warsztatu uczestnik odpowiada na wiadomości od innych uczestników w procedurze ACK.

Po pewnym czasie uczestnik wychodzi z warsztatu i wywołując procedurę V z numerem warsztatu, zwalnia w nim miejsce.
Po przejściu wszystkich wybranych warsztatów uczestnik zwalnia bilet wydarzenia wywołując V(0).

Gdy uczestnik wyjdzie z wydarzenia czeka na innych (aż Pyrkon się zakończy),
a następnie wraca do początku programu i wykonuje go ponownie ad infinitum.

\begin{Verbatim}[numbers=left,xleftmargin=5mm]
while true:
    P(0, num-tickets)
    shuffle workshops
    booked = take max-workshops from workshops
    for w in booked:
        P(w, max-places in w)
        while not timeout:
            ACK()
        V(w)
    V(0)
    JOIN()
\end{Verbatim}

\end{document}

