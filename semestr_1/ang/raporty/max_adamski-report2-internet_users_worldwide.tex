\documentclass[12pt]{article}
\usepackage{lmodern}
\usepackage[top=1in, bottom=1.25in, left=1.25in, right=1.25in]{geometry}
\usepackage[utf8]{inputenc}
\usepackage[T1]{fontenc}
\usepackage[parfill]{parskip}
\pagenumbering{gobble}

\begin{document}

\begin{flushleft} 
	Dariusz Max Adamski \\
	Grupa 2 Informatyka WI\\
	Nr indeksu 136674
\end{flushleft}

\begin{center} 
	\vspace{0.8cm} \Large
	Internet Users Worldwide
	\vspace{0.5cm}
\end{center}

Chart 1 shows the estimated number of people online worldwide, during the period of 1995 to 2002. In 1995 the number of internet users was approximately 26 million. It rose steadily to 76 million in 1997, then continued to rise, at an even quicker pace, to 195 million in 1999. The number of internet users more than doubled to 451 million in the year 2000. Then it rose steadily again, to 552 million in 2001 and to 580 million in 2002.

Chart 2 shows the number of internet users in the world, grouped by region, in 2003. The majority of internet users - 53\% - resided in the United States or Canada. Europe constituted 24\% of the total internet user base, which was approximately half of USA’s and Canada’s. Asia and Pacific accounted for 15\% of internet users, while South America accounted for 6\%. Africa and the Middle Mast had the least internet users - only 1\% each.

In general, chart 1 shows that the number of internet users worldwide increased 22 times over the period of 7 years. Chart 2 illustrates that most of these users reside in the USA, Canada, and to lesser extent Europe and Asia.

199 words

\end{document}
